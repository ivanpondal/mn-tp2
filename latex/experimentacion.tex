% Funcion para poner imagenes que tienen nombre con underscore:
\newcommand{\imagenB}[2]{%
\includegraphics[width=#1\textwidth]{#2}
\endgroup}

\def\imagen{\begingroup
\catcode`\_=12
\imagenB}
% -----------------------------------------

\subsection{Variando el parámetro $c$}\label{exp_gem_2}
Este experimento consite en variar el parámetro $c$ y analizar como impacta esta
variación en los resultados obtenidos al generar el ranking con GeM.

Recordemos que $c$ es el coeficiente de amortiguación, que regula que tanto afecta
el \textit{navegante aleatorio} al resultado final.

Como es intuitivo de pensar, nuestra hipotesis en este experimento es que al tender
$c$ a cero aumenta la influencia del \textit{navegante aleatorio} en el puntaje de
cada página y, por lo tanto, más se parecen los puntajes de todas la páginas.
Es decir, menos dejan de importar los links salientes de las distintas páginas
del grafo original.

Por otro lado, cuando $c$ tiende a uno se debilita la influencia del \textit{navegante
aleatorio}, causando que el puntaje dependa solo de la matriz $H + au^{t}$, descrita
en la sección \ref{sec:gem_model}.

Se muestran a continuación dichos resultados para diferentes valores de $c$:

\begin{table}[H]
    \begin{center}
        \begin{tabular}{| c | l | c |}
            \hline
            Posicion & Equipo & Puntaje \\ \hline
            1 & Vélez Sarsfield & 0.033333 \\
            2 & Unión & 0.033333 \\
            3 & Gimnasia y Esgrima (LP) & 0.033333 \\
            4 & Estudiantes (LP) & 0.033333 \\
            5 & Defensa y Justicia & 0.033333 \\
            6 & Crucero del Norte & 0.033333 \\
            7 & Colón & 0.033333 \\
            \vdots & \quad\vdots & \vdots \\
            24 & Lanús & 0.033333 \\
            25 & San Lorenzo & 0.033333 \\
            26 & Rosario Central & 0.033333 \\
            27 & River Plate & 0.033333 \\
            28 & Racing Club & 0.033333 \\
            29 & Quilmes & 0.033333 \\
            30 & Olimpo & 0.033333 \\
            \hline
        \end{tabular}
        \begin{tabular}{| c | l | c |}
            \hline
            Posicion & Equipo & Puntaje \\ \hline
            1 & Boca Juniors & 0.051301 \\
            2 & San Lorenzo & 0.044563 \\
            3 & River Plate & 0.044200 \\
            4 & Racing Club & 0.040067 \\
            5 & Aldosivi & 0.039451 \\
            6 & Rosario Central & 0.039114 \\
            7 & Quilmes & 0.036699 \\
            \vdots & \quad\vdots & \vdots \\
            24 & Godoy Cruz & 0.028356 \\
            25 & Argentinos Juniors & 0.028208 \\
            26 & Temperley & 0.027904 \\
            27 & Crucero del Norte & 0.027353 \\
            28 & Nueva Chicago & 0.026346 \\
            29 & Atlético de Rafaela & 0.025776 \\
            30 & Colón & 0.025577 \\
            \hline
        \end{tabular}
        \captionsetup{justification=centering}
        \caption{A izquierda: puntajes obtenidos con $c=0$, a derecha: puntajes obtenidos con $c=0.3$}
        \label{exp_resultados_variar_c_1}
    \end{center}
\end{table}

\begin{table}[H]
    \begin{center}
        \begin{tabular}{| c | l | c |}
            \hline
            Posicion & Equipo & Puntaje \\ \hline
            1 & Boca Juniors & 0.086019 \\
            2 & Aldosivi & 0.065353 \\
            3 & River Plate & 0.063500 \\
            4 & San Lorenzo & 0.062035 \\
            5 & Rosario Central & 0.048473 \\
            6 & Racing Club & 0.047878 \\
            7 & San Martín (SJ) & 0.043956 \\
            \vdots & \quad\vdots & \vdots \\
            24 & Godoy Cruz & 0.017516 \\
            25 & Temperley & 0.016045 \\
            26 & Crucero del Norte & 0.016039 \\
            27 & Argentinos Juniors & 0.015398 \\
            28 & Nueva Chicago & 0.014232 \\
            29 & Atlético de Rafaela & 0.011388 \\
            30 & Colón & 0.010276 \\
            \hline
        \end{tabular}
        \begin{tabular}{| c | l | c |}
            \hline
            Posicion & Equipo & Puntaje \\ \hline
            1 & Boca Juniors & 0.095290 \\
            2 & Aldosivi & 0.075151 \\
            3 & River Plate & 0.068142 \\
            4 & San Lorenzo & 0.065265 \\
            5 & Rosario Central & 0.050875 \\
            6 & Racing Club & 0.048824 \\
            7 & San Martín (SJ) & 0.047118 \\
            \vdots & \quad\vdots & \vdots \\
            24 & Godoy Cruz & 0.014148 \\
            25 & Crucero del Norte & 0.013132 \\
            26 & Temperley & 0.012487 \\
            27 & Argentinos Juniors & 0.011286 \\
            28 & Nueva Chicago & 0.011239 \\
            29 & Atlético de Rafaela & 0.007559 \\
            30 & Colón & 0.006003 \\
            \hline
        \end{tabular}
        \captionsetup{justification=centering}
        \caption{A izquierda: puntajes obtenidos con $c=0.85$, a derecha: puntajes obtenidos con $c=1$}
        \label{exp_resultados_variar_c_2}
    \end{center}
\end{table}

Analizando los resultados tenemos que:

\begin{itemize}
    \item Para $c=0$, la tabla de puntajes se condice con la hipótesis planteada.
        Los puntajes de los equipos no solo son parecidos, sino que son iguales.
        Y dicho puntaje es $0.0\hat{3} = 1/30 = 1/n$, dónde $n$ es la cantidad de
        equipos totales.
    \item A medida que el $c$ aumenta (desde $0.3$ a $1$) las posiciones van convergiendo.
        Ya con $c=0.3$, 6 de los 7 primeros equipos aparecen en los primeros 7 puestos con $c=1$,
        y los últimos 7 equipos con $c=0.3$ aparecen en los últimos 7 puestos con $c=1$.
    \item A su vez, la diferencia entre los resultados con $c=85$ y $c=1$ es muy chica:
        de los 14 equipos mostrados en el Cuadro \ref{exp_resultados_variar_c_2} solo dos
        de ellos cambian de posición (Crucero del Norte y Temperley).
\end{itemize}

En base a lo analizado, podemos concluir que los resultados del experimento corroboran
las hipotesis planteadas.
