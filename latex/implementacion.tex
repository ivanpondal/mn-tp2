\subsection{Scripts Rankings Deportivos}

Para los dos modelos siendo estudiados, GeM y el sistema de puntaje de la AFA,
decidimos implementarlos com scripts para \textsc{MATLAB/Octave}, dado que lo que nos
interesaba era analizar los resultados de los mismos y no analizar su tiempo de
cómputo u otro atributo relacionado a su implementación.

\subsubsection{GeM}

El script posee varios parámetros de entrada que pasamos a describir a
continuación:

\begin{itemize}
	\item \textbf{in_filename:} Dirección de archivo con datos de entrada.
	\item \textbf{out_filename:} Dirección de archivo donde escribir el
	resultado final.
	\item \textbf{team_codes_filename:} Archivo opcional con el número de equipo
	asociado a su nombre correspondiente.
	\item \textbf{c:} Parámetro de amortiguación descrito en la
	sección \ref{sec:gem_model} (Por defecto 0.85).
	\item \textbf{date_limit:} Campo opcional con la cantidad de fechas a tomar
	en cuenta (Por defecto toma todas las fechas disponibles).
\end{itemize}

Ahora explicaremos el código desarrollado:

\begin{enumerate}
	\item Leemos la cantidad de equipos y partidos para después cargar todos los
	partidos disponibles y generar nuestra matriz $A$.
	\begin{lstlisting}
	Cargar numero de partidos y equipos
	Crear matriz A de tamanio equipos*equipos llena de 0s
	Mientras partidos > 0
		Cargo numero de fecha del partido junto a los equipos y el resultado
		Si habia limite de fecha y ya se cumplio
			partidos = 0
		Si no
			Si el primer equipo perdio
				A[numero primer equipo][numero segundo equipo] += diferencia de puntos
			Si el segundo equipo perdio
				A[numero segundo equipo][numero primer equipo] += diferencia de puntos
			Fin si
		Fin si
		Decremento partidos en uno
	Fin mientras
	\end{lstlisting}

	\item Generamos la matriz $H$.
	\begin{lstlisting}
	Crear matriz H de tamanio equipos*equipos llena de 0s
	Crear vector a de tamanio equipos lleno de 0s
	Para i de 1 a equipos
		sumaFila = suma total de elementos en fila i de A
		Si la suma > 0
			La fila i de H seran todos los elementos de la fila i de A
			cada uno dividido por sumaFila
		Si no
			Pongo un 1 en la posicion i del vector a
		Fin si
	Fin para
	\end{lstlisting}

	\item Generamos la matriz $G$.
	\begin{lstlisting}
	Crear vector e de tamanio equipos lleno de 1s
	Crear vector u para equipos invictos con todos elementos 1/equipos
	Crear vector v para teletransportacion con todos elementos 1/equipos
	Crear matriz G como resultado de hacer
	G = c*[H + a*transponer(u)] + (1 - c)*e*transponer(v)
	\end{lstlisting}

	\item Busco el autovector asociado al autovalor $\lambda = 1$.
	\begin{lstlisting}
	Generar matrices V y l de autovectores y autovalores de mi matriz G transpuesta
	V es una matriz con autovectores como columnas
	l es una matriz con autovalores en su diagonal
	i = 1
	Mientras valorAbsoluto(l[i][i] - 1) > 0.0001
		Incremento i en una unidad
	Fin mientras
	Crear vector x correspondiente a la columna i de la matriz V
	\end{lstlisting}

	\item Normalizo el vector solución.
	\begin{lstlisting}
	x = valorAbsolutoACadaElemento(x)/sumaElementos(valorAbsolutoACadaElemento(x))
	\end{lstlisting}

	\item Genero matriz de solución.
	\begin{lstlisting}
	Crear matriz S de tamanio equipos*2 y llena de 0s
	Para i de 1 a equipos
		S[i][1] = i
		S[i][2] = x[i]
	Fin para
	\end{lstlisting}

	\item Ordeno y escribo la matriz solución.
	\begin{lstlisting}
	Ordenar crecientemente por segundo elemento de filas la matriz S (ranking)
	Si tengo archivo con nombres de equipos
		Creo mapa teamcodes con cada numero de equipo asociado a su nombre
	Fin si

	Si tengo nombres de equipos cargados
		Para i de 0 a equipos - 1
			Escribo numero, nombre y ranking del equipo numero equipos - i
		Fin para
	Si no
		Para i de 0 a equipos - 1
			Escribo numero y ranking del equipo numero equipos - i
		Fin para
	Fin si
	\end{lstlisting}

\end{enumerate}

\subsubsection{AFA}

