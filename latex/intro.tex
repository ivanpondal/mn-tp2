 El objetivo principal de este Trabajo Práctico es estudiar, implementar y analizar
 algoritmos de Rankeo en dos escenarios distintos: tanto para páginas Web como para
 competencias deportivas.

Comenzaremos haciendo una breve introducción al algoritmo \textit{PageRank}, el cual
es muy conocido por ser utilizado por el buscador Google,
 para luego pasar a describir el Modelo que lo sostiene,
donde explicaremos como resuelve este algoritmo el problema de rankear
páginas web.

Luego, veremos una posible optimización de \textit{PageRank} al modelar
el problema de una manera equivalente utilizando matrices esparsas. Para esto,
presentaremos y analizaremos 3 estructuras de datos distintas que nos permitirán mejorar la
complejidad espacial y temporal de dicho algoritmo. Además, demostraremos que el Algoritmo 1 propuesto en Kamvar et al.\cite{Kamvar2003}
para trabajar con matrices esparsas es correcto.

A continuación, presentaremos el Modelo del algoritmo \textit{GeM} (Generalized Markov chains Method) basado en
\textit{PageRank}. \textit{GeM}, a diferencia de su padre, no se orienta a páginas web, sino a competencias
deportivas, por lo que mostraremos los aspectos en los cuales difieren.
En particular, mostraremos el problema de considerar deportes con empates al utilizar \textit{GeM} y dos formas
alternativas de modelar dicho escenario.

Una vez finalizada la parte del Modelo, pasaremos a describir la Implementación de los
diferentes algoritmos presentados. Dichas implementaciones fueron realizadas algunas en
\texttt{C++} y otras en \texttt{MATLAB/Octave}.

Ya llegando al final, pasaremos a presentar la Experimentación realizada, a la vez
que iremos analizando y discutiendo los resultados obtenidos.

Los experimentos realizados para \textit{PageRank} fueron:
\begin{itemize}
    \item Comparación Page Rank normal vs Optimización matriz esparsa.
    \item Convergencia del algoritmo.
    \item Tiempos de ejecución.
    \item Calidad de los resultados.
\end{itemize}

Y para \textit{GeM}:
\begin{itemize}
    \item Variación del parámetro $c$.
    \item Evolución del ranking por cada iteración.
\end{itemize}

Para finalizar, cerraremos el presente informe con una conclusión, en la cual
discutiremos acerca de los algoritmos vistos, así como de la experimentación realizada.
También, contaremos las dificultades encontradas al realizar el Trabajo Práctico,
las posibles continuaciones que se podrían realizar, y si los objetivos planteados
fueron alcanzados.
