\subsection{Rankings en competencias deportivas}

Elegir un sistema de puntos que sea justo para todos los participantes en un
deporte no es una tarea sencilla. Existen muchos factores que afectan el
resultado de una competencia como lo pueden ser el orden en el que deben
competir entre si los equipos, generando desbalances respecto las capacidades de
cada uno. Es por esto que a continuación presentaremos el modelo GeM
\footnote{Angela Y. Govan, Carl D. Meyer, and Rusell Albright. Generalizing
google’s pagerank to rank national football league teams. In Proceedings of SAS
Global Forum 2008, 2008.} que busca modelar los resultados de forma tal que estos
factores impacten lo menos posible en el posicionamiento final de la tabla de
puntajes.

Con el fin de experimentar con distintos modelos, a lo largo del informe
trabajaremos sobre los resultados del Torneo de Primera División
2015\footnote{Campeonato de Primera División 2015, \textit{Julio H. Grondona} \\
\url{http://www.afa.org.ar/html/9/estadisticas-de-primera-division}}, donde
utilizaremos el sistema de ranking estándar de la AFA como punto de comparación.

\subsubsection{Generalized Markov chains Method (GeM)}

\subsubsection*{Definición del método}

El método GeM es el resultado de tomar el algoritmo PageRank y mediante pequeñas
modificaciones utilizar su potencial para establecer un ranking de equipos.
Análogo a PageRank, los equipos pasan a formar parte de un grafo dirigido
con pesos, donde cada nodo representa un equipo y los pesos de cada arista
reflejan el resultado de los partidos jugados entre los vértices conectados.

~

Formalmente, el modelado se realiza de la siguiente manera:
\begin{enumerate}
	\item Representamos el torneo como un grafo con pesos dirigidos de $n$
	nodos, donde $n$ es igual a la cantidad de equipos que participan. Cada
	equipo tiene su respectivo nodo y las aristas contienen como peso la
	diferencia positiva entre los nodos conectados.

	\item Definimos la matriz de adyacencia $A \in \mathbb{R}^{n \times n}$.
		\begin{equation*}
			A_{ij} =
				\begin{cases}
					w_{ij} & \text{si el equipo $i$ perdió contra $j$}\\
					0 & \text{caso contrario}
				\end{cases}
		\end{equation*}
		Donde $w_{ij}$ es la suma total de diferencia positiva de puntaje sobre todos
		los partidos en los que $i$ perdió contra $j$.

	\item Definimos la matriz $H \in \mathbb{R}^{n \times n}$.
		\begin{equation*}
			H_{ij} =
				\begin{cases}
					A_{ij}/\sum_{k = 1}^{n}A_{ik} & \text{si hay un link de $i$ a $j$}\\
					0 & \text{caso contrario}
				\end{cases}
		\end{equation*}

	\item Definimos la matriz GeM, $G \in \mathbb{R}^{n \times n}$ con $u, v, a, e \in \mathbb{R}^n$ y $c \in \mathbb{R}$.
		\begin{gather*}
			G = c(H + au^{t}) + (1 - c)ev^{t} \\
			\sum_{k = 1}^{n}v_{k} = 1 \qquad \sum_{k = 1}^{n}u_{k} = 1 \qquad \forall_{i = 1..n} e_{i} = 1 \\
			0 \leq c \leq 1 \qquad
			a_{i} =
			\begin{cases}
				1 & \text{si la fila $i$ de $H$ es un vector nulo} \\
				0 & \text{caso contrario}
			\end{cases}
		\end{gather*}

	\item Por último tenemos que el ranking de los equipos estará definido por
		el vector $\pi \in \mathbb{R}^n$ tal que
		\begin{gather*}
			\pi^{t} = \pi^{t}G \\
			\text{o si tomamos la transpuesta en ambos lados} \\
			G^{t}\pi = \pi
		\end{gather*}
\end{enumerate}

De esta forma al igual que con PageRank, calculando el autovector $\pi$
obtenemos nuestro ranking. Este modelo permite cierta flexibidad a partir del
$u$, $v$ y $c$ que tomemos.

El vector de probabilidad $u$ se aplicará en el caso de que un equipo se encuentre invicto, esto
es el equivalente a que en PageRank un sitio no tenga ningún link saliente, por
lo tanto su tratamiento es el mismo, se le asigna a la fila correspondiente el
vector con las probabilidades de saltar a otro nodo. Por lo tanto, el vector $u$
nos permite definir con qué probabilidades un equipo invicto perdería contra el
resto de los participantes. En el caso de PageRank, este es un vector de
distribución uniforme, donde es igual la posibilidad de saltar a cualquiera de
los otros nodos, una posible alternativa sería definirlo como un vector cuyas
probabilidades se basen en algún ranking anterior.

El vector de probabilidad $v$, nos da otro tipo de personalización que es
la del \textit{navegante aleatorio}. Esta es la probabilidad de que un equipo
cualquiera independientemente de los resultados registrados, pierda contra el
resto de los equipos. En PageRank, esto lo veíamos como la posibilidad de que
estando navegando el grafo, uno se \textit{teletransportará} a otro nodo
independientemente de las conexiones de los mismos. Este vector por defecto
también suele tomar el valor de la distribución uniforme.

Por último tenemos nuestro valor $c$ que actua como un factor de amortiguación
donde lo que se modifica es cuánto afecta el \textit{navegante aleatorio} al
resultado final, donde con $c = 0$, únicamente influye el \textit{navegante
aleatorio} y con $c = 1$ se elimina el efecto del mismo.

\subsubsection*{Modelado del empate}

Una particularidad de este sistema que se puede observar en la definición del
mismo es que no contempla los partidos donde hubo empate. Un empate equivale
a que no exista un perdedor y por ende no se modifica el peso de ningún nodo.
Para deportes donde el empate no es algo frecuente esto no sería un problema,
pero si tomamos como ejemplo el fútbol, donde los empates son algo mucho más
común, el ignorar estos partidos afecta notablemente el ranking.
