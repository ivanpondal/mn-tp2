\subsection{Rankings en competencias deportivas}

Elegir un sistema de puntos que sea justo para todos los participantes en un
deporte no es una tarea sencilla. Existen muchos factores que afectan el
resultado de una competencia como lo pueden ser el orden en el que deben
competir entre si los equipos, generando desbalances respecto las capacidades de
cada uno. Es por esto que a continuación presentaremos el modelo GeM
\footnote{Angela Y. Govan, Carl D. Meyer, and Rusell Albright. Generalizing
google’s pagerank to rank national football league teams. In Proceedings of SAS
Global Forum 2008, 2008.} que busca modelar los resultados de forma tal que estos
factores impacten lo menos posible en el posicionamiento final de la tabla de
puntajes.

Con el fin de experimentar con distintos modelos, a lo largo del informe
trabajaremos sobre los resultados del Torneo de Primera División
2015\footnote{Campeonato de Primera División 2015, \textit{Julio H. Grondona} \\
\url{http://www.afa.org.ar/html/9/estadisticas-de-primera-division}}, donde
utilizaremos el sistema de ranking estándar de la AFA como punto de comparación.

\subsubsection{Generalized Markov chains Method (GeM)}
El método GeM es el resultado de tomar el algoritmo PageRank y mediante pequeñas
modificaciones utilizar su potencial para establecer un ranking de equipos.
Análofo a PageRank, los equipos pasan a formar parte de un grafo dirigido
con pesos, donde cada nodo representa un equipo y los pesos de cada arista
reflejan el resultado de los partidos jugados entre los vértices conectados.

~

Formalmente, el modelado se realiza de la siguiente manera:
\begin{enumerate}
	\item Representamos el torneo como un grafo con pesos dirigidos de $n$
	nodos, donde $n$ es igual a la cantidad de equipos que participan. Cada
	equipo tiene su respectivo nodo y las aristas contienen como peso la
	diferencia positiva entre los nodos conectados.

	\item Definimos la matriz de adyacencia $A$.
		\begin{equation*}
			A_{ij} =
				\begin{cases}
					w_{ij} & \text{si el equipo $i$ perdió contra $j$}\\
					0 & \text{caso contrario}
				\end{cases}
		\end{equation*}
		Donde $w_{ij}$ es la suma total de diferencia positiva de puntaje sobre todos
		los partidos en los que $i$ perdió contra $j$.

	\item Definimos la matriz $H$.
		\begin{equation*}
			H_{ij} =
				\begin{cases}
					A_{ij}/\sum_{k = 1}^{n}A_{ik} & \text{si hay un link de $i$ a $j$}\\
					0 & \text{caso contrario}
				\end{cases}
		\end{equation*}

	\item Definimos la matriz GeM $G$.
		\begin{equation*}
			G = c(H + au^{t}) + (1 - c)ev^{t}
		\end{equation*}
		Donde $0 < c < 1$, $v$ es un vector de probabilidad positivo,

\end{enumerate}
